\chapter*{Introducción}
\addcontentsline{toc}{chapter}{Introducción}

El objetivo del trabajo es la implementación de algoritmos de matemática
discreta en Haskell. Los puntos de partida son 
\begin{itemize}
  \item los temas de la asignatura ``Matemática discreta'' (\cite{Cardenas-15a}),
  \item los temas de la asignatura ``Informática'' (\cite{Alonso-15a}),
  \item el capítulo 7 del libro ``Algorithms: A functional programming
    approach'' (\cite{Rabhi+Lapalme-99}) y
  \item el artículo ``Graph theory'' (\cite{Wikipedia-grafos}) de la Wikipedia.  
\end{itemize}

Haskell es un lenguaje de programación funcional desarrollado en los últimos
años por la comunidad de programadores con la intención de usarlo como
instrumento para la enseñanza de programación funcional. La motivación de este
desarrollo es hacer el análisis y diseño de programas más simple y permitir que
los algoritmos sean fácilmente adaptables a otros lenguajes de programación 
funcionales.

En comparación con otros lenguajes de programación imperativos, la sintaxis de 
Haskell permite definir funciones de forma más clara y compacta. En Haskell las
funciones se consideran valores, al mismo nivel que los tipos enteros o cadenas
en cualquier lenguaje. Por ello, al igual que es habitual que en todos los 
lenguajes una función reciba datos de entrada (de tipo entero, flotante, cadena,
etc) y devuelva datos (de los mismos tipos), en los lenguajes funcionales una
función puede recibir como dato de entrada una función y devolver otra función
como salida, que puede ser construida a partir de sus entradas y por operaciones
entre funciones, como la composición. Esta capacidad nos proporciona métodos más
potentes para construir y combinar los diversos módulos de los que se compone
un programa. Por ejemplo, emulando la forma de operar sobre funciones que 
habitualmente se usa en matemáticas.

La matemática discreta consiste en el estudio de las propiedades de los 
conjuntos finitos o infinitos numerables, lo que hace posible su directa
implementación computacional. En particular, la asignatura ``Matemática 
Discreta'' del grado se centra en estudiar propiedades y algoritmos de la 
combinatoria y la teoría de grafos. A lo largo del trabajo implementaré 
en Haskell definiciones y algoritmos con los que ya he tenido una primera 
aproximación teórica durante el grado.

Los dos primeros capítulos del trabajo sientan la base que me permitirá en los
dos siguientes aproximarme a la teoría de grafos. En el primero, hago
una introducción a la teoría de conjuntos y doy dos posibles representaciones
de conjuntos con las que trabajar en Haskell. En el segundo presento los
conceptos de ``relación'' y de ``función'' que, al igual que los conjuntos, son
necesarios para definir aplicaciones de los grafos. Para su redacción me he 
apoyado en la primera parte de la asignatura ``Álgebra Básica'' del grado.

En el tercer capítulo del trabajo hago una introducción a la teoría de grafos.
En primer lugar doy una representación con la que trabajar en Haskell y un
generador de grafos que nos servirá para comprobar propiedades con 
\texttt{QuickCheck}. Seguidamente hago una galería de grafos conocidos y los
utilizo como ejemplos para las definiciones y los algoritmos que doy en las 
secciones siguientes. Las primeras serán algunas deficiones básicas y después
trato los conceptos de ``morfismos'', ``caminos'' y ``conectividad'' en 
grafos.

Por último, en el cuarto capítulo doy una representación de los grafos a través
de la matriz de adyacencia. Implemento algunos de los resultados que Javier 
Franco Galvín expuso en su Trabajo Fin de Grado: \textit{Aspectos algebraicos 
en teoría de grafos} y compruebo que concuerdan con el desarrollo hecho en el 
capítulo anterior.

Me gustaría destacar el trabajo con los diferentes sistemas que he realizado
a lo largo del proyecto y con diferentes programas: 
\itemize{
\item las librerías de Haskell \texttt{Data.List}, \texttt{Data.Matrix} y 
\texttt{Data.QuickCheck},
\item el paquete \textit{Tikz} de \LaTeX{}, mediante el que he dado la 
representación gráfica de los ejemplos de grafos,
\item \textit{GitHub} como sistema de control de versiones,
\item y \textit{Hspec} como sistema de validación de módulos.
}
 
%%% Local Variables: 
%%% mode: latex
%%% TeX-master: "MD_en_Haskell"
%%% End: 
