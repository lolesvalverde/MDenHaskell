\chapter*{Introducción}
\addcontentsline{toc}{chapter}{Introducción}

El objetivo del trabajo es la implementación de algoritmos de Matemática
Discreta en Haskell. Los puntos de partida son 
\begin{itemize}
\item los temas de la asignatura
\href{https://dl.dropboxusercontent.com/u/15420416/tiddly/emptyMD1314.html}
     {Matemática discreta}\
\footnote{\url{https://dl.dropboxusercontent.com/u/15420416/tiddly/emptyMD1314.html}}
(\cite{Cardenas-15a})

\item los temas de la asignatura 
\href{https://www.cs.us.es/~jalonso/cursos/i1m-15}
     {Informática}\
\footnote{\url{{https://www.cs.us.es/~jalonso/cursos/i1m-15}}}
(\cite{Alonso-15a}) 

\item el capítulo 7 del libro 
\href{http://www.iro.umontreal.ca/~lapalme/Algorithms-functional.html}
     {Algorithms: A functional programming approach}\
\footnote{\url{http://www.iro.umontreal.ca/~lapalme/Algorithms-functional.html}}
(\cite{Rabhi+Lapalme-99}) y 

\item el artículo
\href{https://en.wikipedia.org/wiki/Graph_theory}
     {Graph theory}\
\footnote{\url{https://en.wikipedia.org/wiki/Graph_theory}}
(\cite{Wikipedia-grafos}) de la Wikipedia.  

\end{itemize}
 
%%% Local Variables: 
%%% mode: latex
%%% TeX-master: "MD_en_Haskell"
%%% End: 
