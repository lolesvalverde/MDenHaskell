\chapter{Apéndices}

\section {Sistemas utilizados}

El desarrollo de mi Trabajo de Fin de Grado requería de una infraestructura
técnica que he tenido que trabajar antes de comenzar a desarrollar el contenido.
A continuación, voy a nombrar y comentar los sistemas y paquetes que he 
utilizado a lo largo del proyecto. 

\begin{itemize}
  \item \textbf{Ubuntu como sistema operativo.} El primer paso fue instalar
    \textit{Ubuntu} en mi ordenador portátil personal. Para ello, seguí las 
    recomendaciones de mi compañero Eduardo Paluzo, quien ya lo había hecho 
    antes. 

    Primero, me descargué la imagen del sistema \textit{Ubuntu 16.04 LTS} (para 
    procesador de 64 bits) desde la 
    \href{http://www.ubuntu.com/download/desktop}
    {página de descargas de Ubuntu}
    \footnote{\url{http://www.ubuntu.com/download/desktop}} y también la
    herramienta 
    \href{http://www.linuxliveusb.com/}
    {LinuxLive USB Creator}
    \footnote{\url{http://www.linuxliveusb.com/}} 
    que transformaría mi pendrive en una unidad USB Booteable cargada con la
    imagen de Ubuntu. Una vez tuve la unidad USB preparada, procedí a instalar
    el nuevo sistema: apagué el dispositivo y al encenderlo entré en el  
    \texttt{Boot Menu} de la  \texttt{BIOS} del portátil para arrancar desde el
    Pendrive en vez de hacerlo desde el disco duro. 
    Automáticamente, comenzó la instalación de  \textit{Ubuntu} y solo tuve que
    seguir las instrucciones del asistente para montar Ubuntu manteniendo además
    \textit{Windows 10}, que era el sistema operativo con el que había estado 
    trabajando hasta ese momento.
    
    El resultado fue un poco agridulce, pues la instalación de Ubuntu se había
    realizado con éxito, sin embargo, al intentar arrancar  \textit{Windows} 
    desde la nueva \texttt{GRUB}, me daba un error al cargar la imagen del 
    sistema. Después de buscar el error que me aparecía en varios foros, 
    encontré una solución a mi problema: deshabilité el  \texttt{Security Boot} 
    desde la  \texttt{BIOS} y pude volver a arrancar  \textit{Windows 10} con 
    normalidad.

  \item \textbf{\LaTeX{} como sistema de composición de textos.} La distribución
    de \LaTeX{},\textit{Tex Live}, como la mayoría de software que he utilizado,
    la descargué utilizando el \textit{Gestor de Paquetes Synaptic}.
    Anteriormente, sólo había utilizado  \textit{TexMaker} como editor de 
    \LaTeX{} así que fue el primero que descargué. Más tarde, mi tutor José 
    Antonio me sugirió que mejor descargara el paquete \textit{AUCTex}, pues me
    permitiría trabajar con archivos \TeX{} desde el editor \textit{Emacs}, así
    lo hice y es el que he utilizado para redactar el trabajo. Además de los que
    me recomendaba el gestor, solo he tenido que descargarme el paquete 
    \textit{spanish} de \textit{babel} para poder componer el trabajo, pues
    el paquete \textit{Tikz}, que he utilizado para representar los grafos,
    venía incluido en las sugerencias de \textit{Synaptic}.

  \item \textbf{Haskell como lenguaje de programación.} Ya había trabajado 
    anteriormente con este lenguaje en el grado y sabía que sólo tenía que 
    descargarme la plataforma de \textit{Haskell} y podría trabajar con el 
    editor \textit{Emacs}. Seguí las indicaciones que se dan a los estudiantes
    de primer curso en la 
    \href{http://www.cs.us.es/~jalonso/cursos/i1m-15/sistemas.php}
    {página del Dpto. de Ciencias de la Computación e Inteligencia Artificial}
    \footnote{\url{http://www.cs.us.es/~jalonso/cursos/i1m-15/sistemas.php}}
    y me descargué los paquetes \textit{haskell-platform} y 
    \textit{haskell-mode} desde el \textit{Gestor de Paquetes Synaptic}. La 
    versión de la plataforma de \textit{Haskell} con la que he trabajado
    es la \textit{2014.2.0.0.debian2} y la del compilador \textit{GHC} la
    \textit{7.10.3-7}.


  \item \textbf{Dropbox como sistema de almacenamiento compartido.} Ya había 
    trabajado con \textit{Dropbox} en el pasado, así que crear una carpeta 
    compartida con mis tutores no fue ningún problema; sin embargo, al estar
    \textit{Dropbox} sujeto a software no libre, no me resultó tan sencillo
    instalarlo en mi nuevo sistema. En primer lugar, intenté hacerlo 
    directamente desde \textit{Ubuntu Software}, que intentó instalar 
    \textit{Dropbox Nautilus} y abrió dos instalaciones en paralelo. Se quedó 
    colgado el ordenador, así que maté los procesos de instalación activos,
    reinicié el sistema y me descargué directamente el paquete de instalación
    desde la 
    \href{https://www.dropbox.com/es/install?os=lnx}
    {página de descargas de Dropbox}
    \footnote{\url{https://www.dropbox.com/es/install?os=lnx}} y lo ejecuté
    desde la terminal.
  \item \textbf{Git como sistema remoto de control de versiones.} Era la 
    primera vez que trabajaba con \textit{Git} y me costó bastante adaptarme.
    Mi tutor José Antonio y mi compañero Eduardo Paluzo me ayudaron mucho e
    hicieron el proceso de adaptación mucho más ágil. 
    El manual que Eduardo ha redactado y presenta en su Trabajo Fin de Grado
    me ha sido muy útil; es una pequeña guía que cualquier interesado en el
    empleo de \textit{Git} puede utilizar como introducción.
    La instalación no resultó complicada: a través del \textit{Gestor de 
    Paquetes Synaptic} me descargué el paquete \textit{elpa-magit} y todos
    los demás paquetes necesarios para trabajar con \textit{Git} en mi
    portátil usando el editor \textit{Emacs}.
    Eduardo me ayudó a crear el repositorio \textit{MDenHaskell} en la 
    plataforma \textit{GitHub} y clonarlo en mi ordenador personal. Dicho
    repositorio contiene el total del trabajo y se almacena de forma pública.
    La versión de \textit{Magit} que utilizo es la \textit{2.5.0-2}
\end{itemize}

\section{Mapa de decisiones de diseño en conjuntos}

Una vez comenzado el trabajo y habiendo hecho ya varias secciones acerca de la
Teoría de Grafos, se puso de manifiesto la conveniencia de crear nuevos 
capítulos introduciendo conceptos de la Teoría de Conjuntos y las relaciones
que se pueden presentar entre sus elementos. 

Haskell tiene una librería específica para conjuntos: \texttt{Data.Set}, sin
embargo, no es tan intuitiva ni tan eficiente como la de listas 
(\texttt{Data.List}). A lo largo del proyecto he creado un módulo que permite
trabajar con conjuntos utilizando tanto su representación como listas ordenadas
sin elementos repetidos como su representación como listas sin elementos
repetidos. Dicho módulo se llama \texttt{Conjuntos} y lo presento en el primer
capítulo del Trabajo \ref{cap:Conjuntos}.

\section{Mapa de decisiones de diseño en grafos}

Al comienzo del proyecto, la idea era que las primeras representaciones con las
que trabajara fueran las de \textit{grafos como vectores de adyacencia} y 
\textit{grafos como matrices de adyacencia} que se utilizan en Informática en el
primer curso del Grado, con las que ya había trabajado y estaba familiarizada. 

Las definiciones de Informática están pensadas para grafos ponderados (dirigidos
o no según se eligiera), mientras que en Matemática Discreta apenas se usan 
grafos dirigidos o ponderados; por tanto, el primer cambio en la representación 
utilizada fue simplificar las definiciones de modo que solo trabajáramos con
grafos no dirigidos y no ponderados, pero manteniendo las estructuras vectorial
y matricial que mantenían la eficiencia. 

Las representaciones que utilizan \textit{arrays} en \textit{Haskell} son muy
restrictivas, pues solo admiten vectores y matrices que se puedan indexar, lo
que hace muy complicados todos los algoritmos que impliquen algún cambio en los
vértices de los grafos y, además, no permite trabajar con todos los tipos de
vértices que pudiéramos desear. Decidimos volver a cambiar la representación, y
esta vez nos decantamos por la representación de \textit{grafos como listas de
aristas}, perdiendo en eficiencia pero ganando mucho en flexibilidad de
escritura.

Por último mis tutores sugirieron dar una representación de los grafos
mediante sus matrices de adyacencia y comprobar la equivalencia de las
definiciones que ya había hecho en otros módulos con las dadas para esta 
representación. Para ello he trabajado con la librería \texttt{Data.Matrix} 
de \textit{Haskell}.

%%% Local Variables:
%%% mode: latex
%%% TeX-master: "MD_en_Haskell"
%%% End:
