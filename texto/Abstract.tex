\chapter*{Abstract}
\addcontentsline{toc}{chapter}{Abstract}

Discrete mathematics is characterized as the branch of mathematics
dealing with finite and numerable sets. Concepts and notations from
discrete mathematics are useful in studying and describing objects 
and real-life problems. In particular the graph theory has numerous
applications in logistics.


Throughout this project some of the knowlegde adquired from the course
``Matemática Discreta'' will be given a computational implementation.
The code will be written in Haskell language and a free version of it 
will be available in GitHub under the name: \textit{MDenHaskell}.
The work will focus on the graph theory and will provide some
examples and algorithms in order to give an introduction of it and how
it can be implemented in Haskell. 

At first, two chapters will be presented as a gentle reminder of basic
concepts related to the set theory and the relations that can be established
among them. The third chapter will introduce the main topic, graph theory,
with different representations, definitions and examples on graphs. It will
go through aspects such as morphism, connectivity and paths in graphs. Finally
some properties and advantages of working with adjacency matrices will be 
presented in the fourth chapter.

This project leaves the door open for the community of programmers
to continue and improve it. It can be used as a self-learning tool
as well as to make calculations that by hand would be tedious.
